% Created 2019-12-15 Sun 18:34
% Intended LaTeX compiler: pdflatex


\documentclass[11pt]{article}

% enumerations
\usepackage[inline]{enumitem}

% links
\usepackage{hyperref}

% images
\usepackage{graphicx}
\usepackage{subfig}

% document properties. Used with \maketitle
\author{Marco Ieni}
\date{\today}
\title{DOC-ORG: Docker, Org Mode, Latex}

% source code
\usepackage{minted}
\usepackage{xcolor}
\usemintedstyle{paraiso-light}

\author{Marco Ieni}
\date{\today}
\title{DOC-ORG: Docker, Org Mode, Latex}
\begin{document}

\maketitle

\section{header}
\label{sec:org5d4a2fc}
This examples show what you can do with a two lines long \texttt{header.tex}.
To change the aspect of what you see, edit it!
\subsection{subheader}
\label{sec:org0cbb3b2}
this text is a inside subheader.

\section{some code}
\label{sec:orgf13687e}
\begin{verbatim}
int main()
{
  printf("hello world\n");
}
\end{verbatim}
\section{lists}
\label{sec:org2326adc}
\subsection{numerated}
\label{sec:orgc2b2405}
\begin{enumerate}
\item one
\item two
\begin{enumerate}
\item sub-two
\begin{enumerate}
\item sub-sub-two
\end{enumerate}
\end{enumerate}
\item three
\end{enumerate}

\subsection{unnumerated}
\label{sec:orga31669c}
\begin{itemize}
\item One
\begin{itemize}
\item Two
\begin{itemize}
\item Three
\begin{itemize}
\item Four
\end{itemize}
\end{itemize}
\end{itemize}
\end{itemize}

\subsubsection{trivia}
\label{sec:orgfdc1da3}
Doc-org could mean a lot of things, such as:
\begin{itemize}
\item document - organized
\item docker - org mode
\item docile - organist
\end{itemize}
A team of 42 linguists is currently searching for the answer.
\subsection{with description}
\label{sec:org7f41ceb}
\begin{description}
\item[{element1}] description for element one.
\item[{element2}] Very long description for element two, in order to show a
multiline description.
\end{description}

\section{Some math}
\label{sec:org9d8abfe}
You can insert latex equation, like equation \ref{eq:org810bdd9}.
\begin{equation}
\label{eq:org810bdd9}
\phi = \frac{2\pi fD}{c}
\end{equation}
\subsection{Other latex equations}
\label{sec:org629020b}
Equation \ref{eq:orgde44523} reference.

\begin{equation}
\label{eq:orgde44523}
D = \frac{c\phi}{2\pi f}
\end{equation}

\subsubsection{A more complicated equation}
\label{sec:org0d358cf}

\begin{equation}
\Delta TOF_{est} = \frac{k_T TOF}{1+k_T } - 0.5 \frac{\mu_A' - \mu_T'}{1+k_T}.
\end{equation}

\section{tables}
\label{sec:org270fe9c}

look how cute is this simple table:

\begin{center}
\begin{tabular}{rr}
one & two\\
\hline
1 & 2\\
3 & 4\\
\end{tabular}
\end{center}

\begin{table}[htbp]
\caption{\label{tab:org9379397}
You can add captions to your tables}
\centering
\begin{tabular}{lrrrrr}
Anchor & MAE & Correlation & MMA & MAPE & RMSE\\
\hline
a1 & 6.871 & -0.372 & 1.164 & 0.157 & 8.515\\
a2 & 5.275 & 0.329 & 1.127 & 0.117 & 6.332\\
a3 & 5.406 & 0.672 & 1.117 & 0.114 & 6.457\\
\end{tabular}
\end{table}
\end{document}
