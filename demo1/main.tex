\begin{document}
\maketitle
% source: https://www.overleaf.com/learn/latex/Code_listing

\documentclass{article}
\usepackage[utf8]{inputenc}

\usepackage{listings}
\usepackage{xcolor}

%New colors defined below
\definecolor{codegreen}{rgb}{0,0.6,0}
\definecolor{codegray}{rgb}{0.5,0.5,0.5}
\definecolor{codepurple}{rgb}{0.58,0,0.82}
\definecolor{backcolour}{rgb}{0.95,0.95,0.92}

%Code listing style named "mystyle"
\lstdefinestyle{mystyle}{
  backgroundcolor=\color{backcolour},   commentstyle=\color{codegreen},
  keywordstyle=\color{magenta},
  numberstyle=\tiny\color{codegray},
  stringstyle=\color{codepurple},
  basicstyle=\ttfamily\footnotesize,
  breakatwhitespace=false,
  breaklines=true,
  captionpos=b,
  keepspaces=true,
  numbers=left,
  numbersep=5pt,
  showspaces=false,
  showstringspaces=false,
  showtabs=false,
  tabsize=2
}

%"mystyle" code listing set
\lstset{style=mystyle}


% Created 2020-02-12 Wed 21:13
% Intended LaTeX compiler: pdflatex
\documentclass[11pt]{article}
\usepackage[utf8]{inputenc}
\usepackage[T1]{fontenc}
\usepackage{graphicx}
\usepackage{grffile}
\usepackage{longtable}
\usepackage{wrapfig}
\usepackage{rotating}
\usepackage[normalem]{ulem}
\usepackage{amsmath}
\usepackage{textcomp}
\usepackage{amssymb}
\usepackage{capt-of}
\usepackage{hyperref}
\author{root}
\date{\today}
\title{}
\hypersetup{
 pdfauthor={root},
 pdftitle={},
 pdfkeywords={},
 pdfsubject={},
 pdfcreator={Emacs 26.1 (Org mode 9.1.9)}, 
 pdflang={English}}
\begin{document}

\tableofcontents

\section{section}
\label{sec:orgac0d3a1}
This example shows what you can do with a minimal \texttt{header.tex} file.
To change the aspect of what you see, edit it!
\subsection{subsection}
\label{sec:org1e94d51}
This text is a inside subsection. You can add a reference to section \ref{sec:orgac0d3a1}.

\section{some code}
\label{sec:org7de1900}
\begin{verbatim}
int main()
{
  printf("hello world\n");
}
\end{verbatim}
\section{lists}
\label{sec:orgc518b64}
\subsection{numerated}
\label{sec:orgcbd8616}
\begin{enumerate}
\item one
\item two
\begin{enumerate}
\item sub-two
\begin{enumerate}
\item sub-sub-two
\end{enumerate}
\end{enumerate}
\item three
\end{enumerate}

\subsection{unnumerated}
\label{sec:orgcdaf3d7}
\begin{itemize}
\item One
\begin{itemize}
\item Two
\begin{itemize}
\item Three
\begin{itemize}
\item Four
\end{itemize}
\end{itemize}
\end{itemize}
\end{itemize}

\subsubsection{trivia}
\label{sec:org77a5be3}
Doc-org could mean a lot of things, such as:
\begin{itemize}
\item document - organized
\item docker - org mode
\item docile - organist
\end{itemize}
A team of 42 linguists is currently searching for the answer.
\subsection{with description}
\label{sec:org8af2cc0}
\begin{description}
\item[{element1}] description for element one.
\item[{element2}] Very long description for element two, in order to show a
multiline description.
\end{description}

\section{Some math}
\label{sec:org20ec65f}
You can insert latex equation, like equation \ref{eq:org9f3a116}.
\begin{equation}
\label{eq:org9f3a116}
\phi = \frac{2\pi fD}{c}
\end{equation}
\subsection{Other latex equations}
\label{sec:org7a9e077}
Equation \ref{eq:org1f5476e} reference.

\begin{equation}
\label{eq:org1f5476e}
D = \frac{c\phi}{2\pi f}
\end{equation}

\subsubsection{A more complicated equation}
\label{sec:orgbc78261}

\begin{equation}
\Delta TOF_{est} = \frac{k_T TOF}{1+k_T } - 0.5 \frac{\mu_A' - \mu_T'}{1+k_T}.
\end{equation}

\section{tables}
\label{sec:org9142079}

look how cute is this simple table:

\begin{center}
\begin{tabular}{rr}
one & two\\
\hline
1 & 2\\
3 & 4\\
\end{tabular}
\end{center}

\begin{table}[htbp]
\caption{\label{tab:org647e189}
You can add captions to your tables}
\centering
\begin{tabular}{lrrrrr}
Anchor & MAE & Correlation & MMA & MAPE & RMSE\\
\hline
a1 & 6.871 & -0.372 & 1.164 & 0.157 & 8.515\\
a2 & 5.275 & 0.329 & 1.127 & 0.117 & 6.332\\
a3 & 5.406 & 0.672 & 1.117 & 0.114 & 6.457\\
\end{tabular}
\end{table}
\end{document}
-e 
\end{document}
