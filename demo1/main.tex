\documentclass[11pt]{article}

% enumerations
\usepackage[inline]{enumitem}

% links
\usepackage{hyperref}

% images
\usepackage{graphicx}
\usepackage{subfig}

% document properties. Used with \maketitle
\author{Marco Ieni}
\date{\today}
\title{DOC-ORG: Docker, Org Mode, Latex}

% source code
\usepackage{minted}
\usepackage{xcolor}
\usemintedstyle{paraiso-light}


\begin{document}
\maketitle
\section{section}
\label{sec:org88f058c}
This example shows what you can do with a minimal \texttt{header.tex} file.
To change the aspect of what you see, edit it!
\subsection{subsection}
\label{sec:org3551d0b}
This text is a inside subsection. You can add a reference to section \ref{sec:org88f058c}.

\section{some code}
\label{sec:org638502b}
If you specify the programming language, code syntax is highlighted.
\begin{verbatim}
int main()
{
    printf("hello world\n");
}
\end{verbatim}

\begin{verbatim}
# This program prints Hello, world!
print('Hello, world!')
\end{verbatim}

\section{lists}
\label{sec:org61e129b}
\subsection{numerated}
\label{sec:org1f91ed4}
\begin{enumerate}
\item one
\item two
\begin{enumerate}
\item sub-two
\begin{enumerate}
\item sub-sub-two
\end{enumerate}
\end{enumerate}
\item three
\end{enumerate}

\subsection{unnumerated}
\label{sec:org3e4ea35}
\begin{itemize}
\item One
\begin{itemize}
\item Two
\begin{itemize}
\item Three
\begin{itemize}
\item Four
\end{itemize}
\end{itemize}
\end{itemize}
\end{itemize}

\subsubsection{trivia}
\label{sec:org90b5b17}
Doc-org could mean a lot of things, such as:
\begin{itemize}
\item document - organized
\item docker - org mode
\item docile - organist
\end{itemize}
A team of 42 linguists is currently searching for the answer.
\subsection{with description}
\label{sec:org50e7309}
\begin{description}
\item[{element1}] description for element one.
\item[{element2}] Very long description for element two, in order to show a
multiline description.
\end{description}

\section{Some math}
\label{sec:org736e91b}
You can insert latex equation, like equation \ref{eq:orgfb488e2}.
\begin{equation}
\label{eq:orgfb488e2}
\phi = \frac{2\pi fD}{c}
\end{equation}
\subsection{Other latex equations}
\label{sec:org9500a05}
Equation \ref{eq:org4d7d6db} reference.

\begin{equation}
\label{eq:org4d7d6db}
D = \frac{c\phi}{2\pi f}
\end{equation}

\subsubsection{A more complicated equation}
\label{sec:org579094f}

\begin{equation}
\Delta TOF_{est} = \frac{k_T TOF}{1+k_T } - 0.5 \frac{\mu_A' - \mu_T'}{1+k_T}.
\end{equation}

\section{tables}
\label{sec:org579c6ff}

look how cute is this simple table:

\begin{center}
\begin{tabular}{rr}
one & two\\
\hline
1 & 2\\
3 & 4\\
\end{tabular}
\end{center}

\begin{table}[htbp]
\caption{\label{tab:org4d9ac3f}
You can add captions to your tables}
\centering
\begin{tabular}{lrrrrr}
Anchor & MAE & Correlation & MMA & MAPE & RMSE\\
\hline
a1 & 6.871 & -0.372 & 1.164 & 0.157 & 8.515\\
a2 & 5.275 & 0.329 & 1.127 & 0.117 & 6.332\\
a3 & 5.406 & 0.672 & 1.117 & 0.114 & 6.457\\
\end{tabular}
\end{table}

\section{Images}
\label{sec:org589a4f9}
You can insert image, for example see image \ref{fig:orgdd8118c}.

\begin{figure}[htbp]
\centering
\includegraphics[width=.9\linewidth]{./img/example_image.png}
\caption{\label{fig:orgdd8118c}
Example image.}
\end{figure}

\subsection{image matrix}
\label{sec:org96efb8a}
In image \ref{fig:image_matrix} you can find an example of how to combine different images
into one in latex.

\begin{figure}[!tbp]
  \centering
  \subfloat[Case 1.]{\includegraphics[width=0.55\textwidth]{./img/example_image.png}\label{fig:f1}}
  \subfloat[Case 2.]{\includegraphics[width=0.55\textwidth]{./img/example_image.png}\label{fig:f2}}
  \\
  \subfloat[Case 3.]{\includegraphics[width=0.55\textwidth]{./img/example_image.png}\label{fig:f3}}
  \subfloat[Case 4.]{\includegraphics[width=0.55\textwidth]{./img/example_image.png}\label{fig:f4}}
  \\
  \subfloat[Case 5.]{\includegraphics[width=0.55\textwidth]{./img/example_image.png}\label{fig:f5}}
  \subfloat[Case 6.]{\includegraphics[width=0.55\textwidth]{./img/example_image.png}\label{fig:f6}}
  \caption{\label{fig:image_matrix}Combine more pictures into one.}
\end{figure}

\end{document}
