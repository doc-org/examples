% Created 2019-12-15 Sun 14:16
% Intended LaTeX compiler: pdflatex


\documentclass[11pt]{article}

% enumerations
\usepackage[inline]{enumitem}

% links
\usepackage{hyperref}

% images
\usepackage{graphicx}
\usepackage{subfig}

% document properties. Used with \maketitle
\author{Marco Ieni}
\date{\today}
\title{DOC-ORG: Docker, Org Mode, Latex}

% source code
\usepackage{minted}
\usepackage{xcolor}
\usemintedstyle{paraiso-light}

\author{Marco Ieni}
\date{\today}
\title{DOC-ORG: Docker, Org Mode, Latex}
\begin{document}

\maketitle

\section{header}
\label{sec:orgb4ba879}
hello.
\subsection{subheader}
\label{sec:orgf7d8a19}
look how cute is this simple table:

\begin{center}
\begin{tabular}{rr}
one & two\\
\hline
1 & 2\\
3 & 4\\
\end{tabular}
\end{center}

\section{some code}
\label{sec:org57e91e4}
\begin{verbatim}
int main()
{
    printf("hello world\n");
}
\end{verbatim}
\section{lists}
\label{sec:orgdfce148}
\subsection{numerated}
\label{sec:orgf77306c}
\begin{enumerate}
\item one
\item two
\begin{enumerate}
\item sub-two
\begin{enumerate}
\item sub-sub-two
\end{enumerate}
\end{enumerate}
\item three
\end{enumerate}

\subsection{unnumerated}
\label{sec:org6ad293d}
\begin{itemize}
\item One
\begin{itemize}
\item Two
\begin{itemize}
\item Three
\begin{itemize}
\item Four
\end{itemize}
\end{itemize}
\end{itemize}
\end{itemize}

\subsubsection{trivia}
\label{sec:orgddf6258}
Doc-org could mean a lot of things, such as:
\begin{itemize}
\item document - organized
\item docker - org mode
\item docile - organist
\end{itemize}
A team of 42 linguists is currently searching for the answer.
\subsection{with description}
\label{sec:orgb6a5038}
\begin{description}
\item[{element1}] description for element one
\item[{element2}] description for element two
\end{description}
\end{document}
