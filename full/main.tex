\documentclass[11pt]{article}

% enumerations
\usepackage[inline]{enumitem}

% links
\usepackage{hyperref}

% images
\usepackage{graphicx}
\usepackage{subfig}

% document properties. Used with \maketitle
\author{Marco Ieni}
\date{\today}
\title{DOC-ORG: Docker, Org Mode, Latex}

% source code
\usepackage{minted}
\usepackage{xcolor}
\usemintedstyle{paraiso-light}


\begin{document}
\maketitle
\thispagestyle{empty}
\newpage

\pagenumbering{roman}
\tableofcontents
\newpage

\pagestyle{plain}
\pagenumbering{arabic}

\section{Introduction}
\label{sec:org0dbd9ea}
This example shows a lot of what you can achieve with \texttt{doc-org}.

However, \texttt{doc-org} uses latex, emacs and org-mode, so the sky is the limit and
there will never be a \emph{"full"} example.

\subsection{header}
\label{sec:org3f7b1dc}
The \texttt{header.tex} file of this example is a good starting point for your
document, because it supports a lot of features.
Of course, reading all the other examples related to each single feature is
recommended to understand what's the role of every line of the header.
\section{some code}
\label{sec:orgcc639d0}
If you specify the programming language, code syntax is highlighted.
\begin{minted}[]{c}
int main()
{
    printf("hello world\n");
}
\end{minted}

\begin{minted}[]{python}
# This program prints Hello, world!
print('Hello, world!')
\end{minted}

\section{Some math}
\label{sec:orgec30830}
You can insert latex equation, like equation \ref{eq:orgff9a1ae}.
\begin{equation}
\label{eq:orgff9a1ae}
\phi = \frac{2\pi fD}{c}
\end{equation}
\subsection{Other latex equations}
\label{sec:orgf70d531}
Equation \ref{eq:org1b330f9} reference.

\begin{equation}
\label{eq:org1b330f9}
D = \frac{c\phi}{2\pi f}
\end{equation}

\subsubsection{A more complicated equation}
\label{sec:org1cb6d3a}

\begin{equation}
\Delta TOF_{est} = \frac{k_T TOF}{1+k_T } - 0.5 \frac{\mu_A' - \mu_T'}{1+k_T}.
\end{equation}

\section{lists}
\label{sec:org9f368d5}
\subsection{numerated}
\label{sec:orgd76d9a8}
\begin{enumerate}
\item one
\item two
\begin{enumerate}
\item sub-two
\begin{enumerate}
\item sub-sub-two
\end{enumerate}
\end{enumerate}
\item \label{org3efee20}three
\end{enumerate}

Here we refer to item \ref{org3efee20}.

\subsection{unnumerated}
\label{sec:org3265ed4}
\begin{itemize}
\item One
\begin{itemize}
\item Two
\begin{itemize}
\item Three
\begin{itemize}
\item Four
\end{itemize}
\item Five
\end{itemize}
\end{itemize}
\end{itemize}

\subsubsection{trivia}
\label{sec:org332c57b}
Doc-org could mean a lot of things, such as:
\begin{itemize}
\item document - organized
\item docker - org mode
\item docile - organist
\end{itemize}
A team of 42 linguists is currently searching for the answer.

\subsection{Enumeration example}
\label{sec:orgc524146}
\begin{description}
\item[{element1}] description for element one.
\item[{element2}] Very long description for element two, in order to show a
multiline description.
\end{description}
\section{tables}
\label{sec:org82ee4b9}

look how cute is this simple table:

\begin{center}
\begin{tabular}{rr}
one & two\\
\hline
1 & 2\\
3 & 4\\
\end{tabular}

\end{center}

\begin{table}[htbp]
\centering
\begin{tabular}{lrrrrr}
Anchor & MAE & Correlation & MMA & MAPE & RMSE\\
\hline
a1 & 6.871 & -0.372 & 1.164 & 0.157 & 8.515\\
a2 & 5.275 & 0.329 & 1.127 & 0.117 & 6.332\\
a3 & 5.406 & 0.672 & 1.117 & 0.114 & 6.457\\
\end{tabular}
\caption{\label{tab:org82d9f0a}
You can add captions to your tables}

\end{table}

\section{Images}
\label{sec:org274cbe3}
You can insert image, for example see image \ref{fig:orgdb3d32a}.

\begin{figure}[htbp]
\centering
\includegraphics[width=.9\linewidth]{./img/example_image.png}
\caption{\label{fig:orgdb3d32a}
Example image.}
\end{figure}

\subsection{image matrix}
\label{sec:org617cf24}
In image \ref{fig:image_matrix} you can find an example of how to combine different images
into one in latex.

\begin{figure}[!tbp]
  \centering
  \subfloat[Case 1.]{\includegraphics[width=0.55\textwidth]{./img/example_image.png}\label{fig:f1}}
  \subfloat[Case 2.]{\includegraphics[width=0.55\textwidth]{./img/example_image.png}\label{fig:f2}}
  \\
  \subfloat[Case 3.]{\includegraphics[width=0.55\textwidth]{./img/example_image.png}\label{fig:f3}}
  \subfloat[Case 4.]{\includegraphics[width=0.55\textwidth]{./img/example_image.png}\label{fig:f4}}
  \\
  \subfloat[Case 5.]{\includegraphics[width=0.55\textwidth]{./img/example_image.png}\label{fig:f5}}
  \subfloat[Case 6.]{\includegraphics[width=0.55\textwidth]{./img/example_image.png}\label{fig:f6}}
  \caption{\label{fig:image_matrix}Combine more pictures into one.}
\end{figure}

\end{document}
