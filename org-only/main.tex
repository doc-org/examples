% Created 2020-02-15 Sat 13:53
% Intended LaTeX compiler: pdflatex
\documentclass[11pt]{article}
\usepackage[utf8]{inputenc}
\usepackage[T1]{fontenc}
\usepackage{graphicx}
\usepackage{grffile}
\usepackage{longtable}
\usepackage{wrapfig}
\usepackage{rotating}
\usepackage[normalem]{ulem}
\usepackage{amsmath}
\usepackage{textcomp}
\usepackage{amssymb}
\usepackage{capt-of}
\usepackage{hyperref}
\author{Marco Ieni}
\date{\today}
\title{Org mode only}
\hypersetup{
 pdfauthor={Marco Ieni},
 pdftitle={Org mode only},
 pdfkeywords={},
 pdfsubject={},
 pdfcreator={Emacs 26.1 (Org mode 9.1.9)}, 
 pdflang={English}}
\begin{document}

\maketitle
\tableofcontents


\section{Introduction}
\label{sec:org13e3a34}
This is an example of how you can use the \texttt{auto\_latex} feature.
In this feature, in fact, \texttt{header.tex} is not included automatically by
doc-org.
In fact, the latex output of this file will be the same of the emacs
default export.

\section{Advantages}
\label{sec:org94b60f0}
If you don't care about specific latex customization, emacs will do the work for
you.

\section{Disadvantages}
\label{sec:orge9bcaa4}
In my opinion, if you don't like emacs defaults, it becomes complicated to
customize latex output.

\section{Customize latex}
\label{sec:orgff5fa4e}
Here is an example of how you can customize latex output
\begin{verbatim}
# do not show table of content
#+OPTIONS: toc:nil

# custom latex class
#+LaTeX_CLASS: book

# include an entire latex file
#+LATEX_HEADER: \input{my_file.tex}

# extra packages
#+LATEX_HEADER: \usepackage{my-package}
\end{verbatim}

Be sure to check \href{https://orgmode.org/manual/LaTeX-Export.html}{latex export} documentation in org-mode website.

\section{references}
\label{sec:org30103fd}
You can add references to other sections, like section \ref{sec:org13e3a34}.
\end{document}
