% source: https://www.overleaf.com/learn/latex/Code_listing

\documentclass{article}
\usepackage[utf8]{inputenc}

\usepackage{listings}
\usepackage{xcolor}

%New colors defined below
\definecolor{codegreen}{rgb}{0,0.6,0}
\definecolor{codegray}{rgb}{0.5,0.5,0.5}
\definecolor{codepurple}{rgb}{0.58,0,0.82}
\definecolor{backcolour}{rgb}{0.95,0.95,0.92}

%Code listing style named "mystyle"
\lstdefinestyle{mystyle}{
  backgroundcolor=\color{backcolour},   commentstyle=\color{codegreen},
  keywordstyle=\color{magenta},
  numberstyle=\tiny\color{codegray},
  stringstyle=\color{codepurple},
  basicstyle=\ttfamily\footnotesize,
  breakatwhitespace=false,
  breaklines=true,
  captionpos=b,
  keepspaces=true,
  numbers=left,
  numbersep=5pt,
  showspaces=false,
  showstringspaces=false,
  showtabs=false,
  tabsize=2
}

%"mystyle" code listing set
\lstset{style=mystyle}


\begin{document}
\section{Introduction}
\label{sec:orgaf90c21}
This example shows what you can do with a one line \texttt{header.tex} file.

\subsection{subsection}
\label{sec:org2e1c7ce}
This text is a inside subsection.

\section{some code}
\label{sec:org74155f5}
Even if you specify the programming language, code syntax is not highlighted
by default.
\begin{verbatim}
int main()
{
    printf("hello world\n");
}
\end{verbatim}

\begin{verbatim}
# This program prints Hello, world!
print('Hello, world!')
\end{verbatim}

\section{lists}
\label{sec:org431102b}
\subsection{numerated}
\label{sec:org293d61a}
\begin{enumerate}
\item one
\item two
\begin{enumerate}
\item sub-two
\begin{enumerate}
\item sub-sub-two
\end{enumerate}
\end{enumerate}
\item three
\end{enumerate}

\subsection{unnumerated}
\label{sec:orga0f0696}
\begin{itemize}
\item One
\begin{itemize}
\item Two
\begin{itemize}
\item Three
\begin{itemize}
\item Four
\end{itemize}
\end{itemize}
\end{itemize}
\end{itemize}

\subsubsection{trivia}
\label{sec:org355d0b3}
Doc-org could mean a lot of things, such as:
\begin{itemize}
\item document - organized
\item docker - org mode
\item docile - organist
\end{itemize}
A team of 42 linguists is currently searching for the answer.

\section{Some math}
\label{sec:orgf847ceb}
You can insert latex equation, like equation \ref{eq:org079b18e}.
\begin{equation}
\label{eq:org079b18e}
\phi = \frac{2\pi fD}{c}
\end{equation}
\subsection{Other latex equations}
\label{sec:orgc2eb395}
Equation \ref{eq:org484aea1} reference.

\begin{equation}
\label{eq:org484aea1}
D = \frac{c\phi}{2\pi f}
\end{equation}

\subsubsection{A more complicated equation}
\label{sec:org837c08b}

\begin{equation}
\Delta TOF_{est} = \frac{k_T TOF}{1+k_T } - 0.5 \frac{\mu_A' - \mu_T'}{1+k_T}.
\end{equation}

\end{document}
