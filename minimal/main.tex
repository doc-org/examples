\documentclass[11pt]{article}

% enumerations
\usepackage[inline]{enumitem}

% links
\usepackage{hyperref}

% images
\usepackage{graphicx}
\usepackage{subfig}

% document properties. Used with \maketitle
\author{Marco Ieni}
\date{\today}
\title{DOC-ORG: Docker, Org Mode, Latex}

% source code
\usepackage{minted}
\usepackage{xcolor}
\usemintedstyle{paraiso-light}


\begin{document}
\section{Introduction}
\label{sec:org2e17d80}
This example shows what you can do with a one line \texttt{header.tex} file.

\subsection{subsection}
\label{sec:orgae5ba53}
This text is a inside subsection.

\section{some code}
\label{sec:orgcea8fca}
Even if you specify the programming language, code syntax is not highlighted
by default.
\begin{verbatim}
int main()
{
    printf("hello world\n");
}
\end{verbatim}

\begin{verbatim}
# This program prints Hello, world!
print('Hello, world!')
\end{verbatim}

\section{lists}
\label{sec:orgf8c778b}
\subsection{numerated}
\label{sec:orgf74174b}
\begin{enumerate}
\item one
\item two
\begin{enumerate}
\item sub-two
\begin{enumerate}
\item sub-sub-two
\end{enumerate}
\end{enumerate}
\item three
\end{enumerate}

\subsection{unnumerated}
\label{sec:org3940baf}
\begin{itemize}
\item One
\begin{itemize}
\item Two
\begin{itemize}
\item Three
\begin{itemize}
\item Four
\end{itemize}
\end{itemize}
\end{itemize}
\end{itemize}

\subsubsection{trivia}
\label{sec:org9f9fa8b}
Doc-org could mean a lot of things, such as:
\begin{itemize}
\item document - organized
\item docker - org mode
\item docile - organist
\end{itemize}
A team of 42 linguists is currently searching for the answer.

\section{Some math}
\label{sec:orgf9fd7df}
You can insert latex equation, like equation \ref{eq:org6a8f433}.
\begin{equation}
\label{eq:org6a8f433}
\phi = \frac{2\pi fD}{c}
\end{equation}

As you can see, reference to equations works by default, but see \texttt{references}
example if you want to know more.

\subsection{Other latex equations}
\label{sec:org39bc917}
Equation \ref{eq:org71d83c0} reference.

\begin{equation}
\label{eq:org71d83c0}
D = \frac{c\phi}{2\pi f}
\end{equation}

\subsubsection{A more complicated equation}
\label{sec:orgcde0ad4}

\begin{equation}
\Delta TOF_{est} = \frac{k_T TOF}{1+k_T } - 0.5 \frac{\mu_A' - \mu_T'}{1+k_T}.
\end{equation}

\end{document}
