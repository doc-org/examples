\documentclass[11pt]{article}

% enumerations
\usepackage[inline]{enumitem}

% links
\usepackage{hyperref}

% images
\usepackage{graphicx}
\usepackage{subfig}

% document properties. Used with \maketitle
\author{Marco Ieni}
\date{\today}
\title{DOC-ORG: Docker, Org Mode, Latex}

% source code
\usepackage{minted}
\usepackage{xcolor}
\usemintedstyle{paraiso-light}


\begin{document}
\section{Introduction}
\label{sec:orgbe751b9}
This example shows what you can do with a one line \texttt{header.tex} file.

\subsection{subsection}
\label{sec:orge9b704d}
This text is a inside subsection. You can refer to other sections, like section \ref{sec:orgbe751b9}.

If you want references to be clickable add \texttt{\textbackslash{}usepackage\{hyperref\}} to \texttt{header.tex}.
See \texttt{links} example, too.

\section{some code}
\label{sec:org1af2d50}
Even if you specify the programming language, code syntax is not highlighted
by default.

\begin{verbatim}
1  int main()
2  {
3      printf("hello world\n");
\end{verbatim}

In line 3 you print \texttt{hello world} to \texttt{stdout}.
\begin{verbatim}
4      int a = 1 + 2;
5  }
\end{verbatim}

In line 4 you just do an assignment.

In my experience unserscore does not work with code references. For example
\texttt{print\_c} will not work in this case.

\subsection{python code}
\label{sec:orgaf29285}
\begin{verbatim}
# This program prints Hello, world!
print('Hello, world!')
\end{verbatim}

\section{lists}
\label{sec:org8612228}
\subsection{numerated}
\label{sec:org7be2f99}
\begin{enumerate}
\item one
\item two
\begin{enumerate}
\item sub-two
\begin{enumerate}
\item sub-sub-two
\end{enumerate}
\end{enumerate}
\item three
\end{enumerate}

\subsection{unnumerated}
\label{sec:org71ee39b}
\begin{itemize}
\item One
\begin{itemize}
\item Two
\begin{itemize}
\item Three
\begin{itemize}
\item Four
\end{itemize}
\end{itemize}
\end{itemize}
\end{itemize}

\subsubsection{trivia}
\label{sec:org84b63b7}
Doc-org could mean a lot of things, such as:
\begin{itemize}
\item document - organized
\item docker - org mode
\item docile - organist
\end{itemize}
A team of 42 linguists is currently searching for the answer.

\section{Some math}
\label{sec:org8f0a1c1}
You can insert latex equation, like equation \ref{eq:orgb5b7528}.
\begin{equation}
\label{eq:orgb5b7528}
\phi = \frac{2\pi fD}{c}
\end{equation}

As you can see, reference to equations works by default, but see \texttt{references}
example if you want links to be clickable.

\subsection{Other latex equations}
\label{sec:org07a8827}
Equation \ref{eq:org412e0b4} reference.

\begin{equation}
\label{eq:org412e0b4}
D = \frac{c\phi}{2\pi f}
\end{equation}

\subsubsection{A more complicated equation}
\label{sec:orga82ef22}

\begin{equation}
\Delta TOF_{est} = \frac{k_T TOF}{1+k_T } - 0.5 \frac{\mu_A' - \mu_T'}{1+k_T}.
\end{equation}

\end{document}
