\documentclass[11pt]{article}

% enumerations
\usepackage[inline]{enumitem}

% links
\usepackage{hyperref}

% images
\usepackage{graphicx}
\usepackage{subfig}

% document properties. Used with \maketitle
\author{Marco Ieni}
\date{\today}
\title{DOC-ORG: Docker, Org Mode, Latex}

% source code
\usepackage{minted}
\usepackage{xcolor}
\usemintedstyle{paraiso-light}


\begin{document}
\section{Introduction}
\label{sec:org71d4cdd}
This example shows how links and references works.
Every link is clickable!

\section{Sections}
\label{sec:org4f0fa31}
You can add a reference to section \ref{sec:org71d4cdd}.

\section{Equations}
\label{sec:orgfe99674}
You can refer to equation \ref{eq:org67063cd}.
\begin{equation}
\label{eq:org67063cd}
\phi = \frac{2\pi fD}{c}
\end{equation}

\section{Tables}
\label{sec:org0372b85}
Reference to table \ref{tab:org7393be8}.
Caption is required for the reference to work properly.

\begin{table}[htbp]
\caption{\label{tab:org7393be8}
a nice table}
\centering
\begin{tabular}{lr}
Anchor & MAE\\
\hline
a1 & 6.871\\
a2 & 5.275\\
a3 & 5.406\\
\end{tabular}
\end{table}

\section{items}
\label{sec:org07d7b51}
\begin{enumerate}
\item one item
\item \label{org321dbbc}another item
\end{enumerate}
Here we refer to item \ref{org321dbbc}.

\section{links}
\label{sec:orgec9f7ff}
Visit the \href{https://orgmode.org/}{official org mode website}.

\section{manual link}
\label{sec:orgef71ade}
It's also possible to link directly any word
or \hyperlink{thesentence}{any sentence} in you document.

If you read this text, you will get no information.  Really?
Is there no information?

For instance \hypertarget{thesentence}{this sentence}.

\end{document}
