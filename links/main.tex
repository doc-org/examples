% source: https://www.overleaf.com/learn/latex/Code_listing

\documentclass{article}
\usepackage[utf8]{inputenc}

\usepackage{listings}
\usepackage{xcolor}

%New colors defined below
\definecolor{codegreen}{rgb}{0,0.6,0}
\definecolor{codegray}{rgb}{0.5,0.5,0.5}
\definecolor{codepurple}{rgb}{0.58,0,0.82}
\definecolor{backcolour}{rgb}{0.95,0.95,0.92}

%Code listing style named "mystyle"
\lstdefinestyle{mystyle}{
  backgroundcolor=\color{backcolour},   commentstyle=\color{codegreen},
  keywordstyle=\color{magenta},
  numberstyle=\tiny\color{codegray},
  stringstyle=\color{codepurple},
  basicstyle=\ttfamily\footnotesize,
  breakatwhitespace=false,
  breaklines=true,
  captionpos=b,
  keepspaces=true,
  numbers=left,
  numbersep=5pt,
  showspaces=false,
  showstringspaces=false,
  showtabs=false,
  tabsize=2
}

%"mystyle" code listing set
\lstset{style=mystyle}


\begin{document}
\section{Introduction}
\label{sec:org71d4cdd}
This example shows how links and references works.
Every link is clickable!

\section{Sections}
\label{sec:org4f0fa31}
You can add a reference to section \ref{sec:org71d4cdd}.

\section{Equations}
\label{sec:orgfe99674}
You can refer to equation \ref{eq:org67063cd}.
\begin{equation}
\label{eq:org67063cd}
\phi = \frac{2\pi fD}{c}
\end{equation}

\section{Tables}
\label{sec:org0372b85}
Reference to table \ref{tab:org7393be8}.
Caption is required for the reference to work properly.

\begin{table}[htbp]
\caption{\label{tab:org7393be8}
a nice table}
\centering
\begin{tabular}{lr}
Anchor & MAE\\
\hline
a1 & 6.871\\
a2 & 5.275\\
a3 & 5.406\\
\end{tabular}
\end{table}

\section{items}
\label{sec:org07d7b51}
\begin{enumerate}
\item one item
\item \label{org321dbbc}another item
\end{enumerate}
Here we refer to item \ref{org321dbbc}.

\section{links}
\label{sec:orgec9f7ff}
Visit the \href{https://orgmode.org/}{official org mode website}.

\section{manual link}
\label{sec:orgef71ade}
It's also possible to link directly any word
or \hyperlink{thesentence}{any sentence} in you document.

If you read this text, you will get no information.  Really?
Is there no information?

For instance \hypertarget{thesentence}{this sentence}.

\end{document}
